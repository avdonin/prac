\subsection*{Определение константы диссоциации азотной кислоты методом КР}

\vspace{.5em}
\begin{center}
    Иван С. Авдонин
\end{center}

\thispagestyle{empty}

\noindent\textbf{Цель работы: } Интерпретация спектров КР раствора азотной кислоты и расчёт константы диссоциации.

\begin{figure}[h]
\centering
    \begin{tikzpicture}
    \begin{axis}[%
    width = \linewidth,
    height = 0.5\linewidth,
    axis x line*=bottom,
    axis y line*=left,
    %axis line style = {line width=0.1pt},
    %xticklabels = \empty,
    %yticklabels = \empty,
    %xtick = \empty,
    ymax = 2,
    ytick = \empty,
    xlabel shift = 0 pt,
    ylabel shift = 0 pt,
    ylabel = {Relative Raman Intensity}, % Set the labels
    xlabel = {cm$^{-1}$},
    %ylabel style = {rotate=-90,at={(axis description cs:0,.9)}},
    %xlabel style = {at={(axis description cs:1.1,0)}}
    ]
    \addplot [
    %only marks,
    patch type=quadratic spline,
    %mark size=.5,
    blue,mark=none,pattern=horizontal line light blue, domain=500:1300, samples=1000] table {C0.1norm.dat};
    
    \addplot [
    %only marks,
    patch type=quadratic spline,
    %mark size=.5,
    red,mark=none,pattern=horizontal line light red, domain=500:1300, samples=1000] table {C14norm.dat};
 
    \node[
        coordinate,
        pin = {[rotate=90]right:$\delta_{rock}(O-N-O)$}
        ] at (axis cs:638,0.6) { };

    \node[
        coordinate,
        pin = {[rotate=90]right:$\delta_{sciss}(O-N-O)$}
        ] at (axis cs:688,0.6) { };
        
    \node[
        coordinate,
        pin = {[rotate=90]right:$\nu(HO-NO_2)$}
        ] at (axis cs:954,0.8) { };
        
    \node[
        coordinate,
        pin = {4:$\nu(NO_3^-)$}
        ] at (axis cs:1048,1.35) { };
        
    \node[
        coordinate,
        pin = {[rotate=90]right:$\nu(O-N-O)$}
        ] at (axis cs:1305,.9) { };
        
                
    \end{axis}
    \end{tikzpicture}
    \caption{Спектры КР водных растворов 0.1 M (синий) и 14 М (красный) азотной кислоты. 
    \\$\nu$ -- валентные колебания, $\delta$ -- деформационные.}
    \label{fig:cf3cocccl+cpd.IRC}
\end{figure}

\begin{figure}[h!]
\begin{minipage}[h]{.45\linewidth}
\centering
    \begin{tikzpicture}
    \begin{axis}[%
    width = 0.9\linewidth,
    %height = 0.7\linewidth,
    axis x line*=bottom,
    axis y line*=left,
    %axis line style = {line width=0.1pt},
    %xticklabels = \empty,
    %yticklabels = \empty,
    %xtick = \empty,
    %ytick = \empty,
    xlabel shift = 0 pt,
    ylabel shift = 0 pt,
    ylabel = {Peak height}, % Set the labels
    xlabel = {C, mol/L},
    %ylabel style = {rotate=-90,at={(axis description cs:0,.9)}},
    %xlabel style = {at={(axis description cs:1.1,0)}}
    ]
    \addplot[
    only marks,
    %mark size=.5,
    %draw=blue,pattern=horizontal lines light blue] table [mark=*,x index=0,y index=1]{f.dat};
    draw=blue,pattern=horizontal lines light blue] table [mark=*,x index=0,y index=1]{c.dat};
    \addplot[draw=red,pattern=horizontal lines light blue,domain=0:.85, samples=100]{7942.37*x};
    
    \end{axis}
    \end{tikzpicture}
    \caption{Калибровочный график высоты пика от концентрации кислоты.}
    \label{fig:cf3cocccl+cpd.IRC}
\end{minipage}
\hspace{10pt}
\begin{minipage}[h]{.45\linewidth}
\centering
    \begin{tikzpicture}
    \begin{axis}[%
    width = 0.9\linewidth,
    %height = 0.7\linewidth,
    axis x line*=bottom,
    axis y line*=left,
    %axis line style = {line width=0.1pt},
    %xticklabels = \empty,
    %yticklabels = \empty,
    %xtick = \empty,
    %ytick = \empty,
    xlabel shift = 0 pt,
    ylabel shift = 0 pt,
    ylabel = {$lgK_C$}, % Set the labels
    xlabel = {C, mol/L},
    %ylabel style = {rotate=-90,at={(axis description cs:0,.9)}},
    %xlabel style = {at={(axis description cs:1.1,0)}}
    ]
    \addplot[
    only marks,
    %mark size=.5,
    %draw=blue,pattern=horizontal lines light blue] table [mark=*,x index=0,y index=1]{f.dat};
    draw=blue,pattern=horizontal lines light blue] table [mark=*,x index=0,y index=1]{ClgK.dat};
    \addplot[draw=red,pattern=horizontal lines light blue,domain=0:13, samples=100]{-0.1446*x+1.4768};
    
    \end{axis}
    \end{tikzpicture}
    \caption{$lgK_C = -0.145C+1.477$ \\$R^2 = 0.998$}
    \label{fig:cf3cocccl+cpd.IRC}
    \end{minipage}
\end{figure}

\begin{equation*}
    K_a = \lim_{C_{HNO_3} \to 0} lgK_C \approx 10^{1.477} \approx 30
\end{equation*}
